% Preamble
\documentclass[a4paper]{scrartcl}

% Packages
\usepackage[utf8]{inputenc}
\usepackage[dutch]{babel}
\usepackage{enumerate}

% Meta
\title{DS Huiswerk 1}
\author{Luuk~Berkers~6793592 \and Timo Dijkstra~studentnummer}
\date{\today}

% Document
\begin{document}
    \maketitle
    
    \section{Pannekoeken}

    \section{Kvick S\"ort}
    \begin{enumerate}[(a)]
        \item Een gerandomiseerde pivotkeuze.
        Dat is te zien aan de dobbelsteen.
        \item Uitgaande van een gerandomiseerde versie van de implementatie van Cormen et al.:
        \begin{verbatim}
algorithm partition(A, lo, hi) is
    pivot := A[random(lo, hi)]
    i := lo
    for j := lo to hi do
        if A[j] < pivot then
            swap A[i] with A[j]
            i := i + 1
    swap A[i] with A[hi]
    return i
        \end{verbatim}
        Te zien is dat er een vergelijking plaatsvind voor elke iteratie van de for-loop.
        Het aantal vergelijkingen is dus \(hi - lo + 1\).
        Het algorithme wordt voor de eerste partitionering aangeroepen als
        \verb|partition(A, 0, length(A) - 1)|, dus het aantal vergelijkingen is dan
        \((n - 1) - 0 + 1 = n\).
        \item Als je kunt garanderen dat de blokjes hun onderlinge volgorde behouden heb je een
        stabiel sorteeralgoritme.
        Wat Quicksort eigenlijk juist niet is.
        \item Zolang je begrijpt wat pijlen aangeven en wat een dobbelsteen doet wel.
        Het is nog steeds wel complexe materie die niet iedereen aan de hand van enkel illustraties
        zal begrijpen.
    \end{enumerate}

    \section{Optellen}

    \section{De Grote Omega}

    \section{Een zware klus}
\end{document}
